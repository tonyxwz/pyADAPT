\hypertarget{pyadapt-tutorial}{%
\section{PyADAPT Tutorial}\label{pyadapt-tutorial}}

ADAPT was originally written in MATLAB and now we at CBio group would
like to write it in python a.k.a, pyADAPT. There are mainly three
classes in the python package, \texttt{DataSet}, \texttt{Model}, and
\texttt{ADAPT}. They are organized in basically the same way as the
MATLAB version.

\hypertarget{getting-started}{%
\subsection{Getting started}\label{getting-started}}

\hypertarget{python}{%
\subsubsection{Python}\label{python}}

To use pyADAPT, one should have some basic knowledge about python
programming. Python is a dynamic programming language and is very
similar to MATLAB (in which ADAPT was written originally). These are
some good resources to learn python. I would recommend learning list
comprehension, lambda expression and object oriented programming (class
definition) \textbf{besides} basic syntax.

\begin{itemize}
\tightlist
\item
  lambda expression:

  \begin{itemize}
  \tightlist
  \item
    used a lot in ODE solvers and optimizers
  \item
    \url{https://docs.python.org/3/reference/expressions.html\#lambda}
  \end{itemize}
\item
  object oriented programming:

  \begin{itemize}
  \tightlist
  \item
    how to define a model for example the toy model
  \item
    \url{https://python.swaroopch.com/oop.html}
  \end{itemize}
\item
  list comprehension

  \begin{itemize}
  \tightlist
  \item
    used a lot in this project for simplicity
  \item
    \url{https://www.python-course.eu/python3_list_comprehension.php}
  \end{itemize}
\end{itemize}

\hypertarget{scipy-suite}{%
\subsubsection{Scipy suite}\label{scipy-suite}}

Python is free and open source. Unlike MATLAB, which is built to do
scientific computing, python is a general purpose programming language.
One can use it to write HTTP servers, image processing apps or instant
messengers. But researchers are more interested in using python to do
what MATLAB could do. Luckily, there's scipy suite for this purpose. It
provides an alternative for almost everything in MATLAB. To make this
tutorial more specific to pyADAPT, only those tools used will be
covered.

\begin{itemize}
\tightlist
\item
  matplotlib: plotting in python
\item
  scipy.integrate.solve\_ivp: \texttt{ode15s} in python, solve a initial
  value problem
\item
  scipy.optimize.least\_squares: \texttt{lsqnonlin} in python, optimize
  the parameters using Levenberg--Marquardt algorithm. I am not using it
  directly but through a wrapper called \texttt{lmfit.minimize}.
\end{itemize}

It is useful to check out these third-party libs briefly before writing
you own model.

\hypertarget{pyadapt}{%
\subsubsection{pyADAPT}\label{pyadapt}}

Either you get the source code of \texttt{pyADAPT} by downloading an
archive or by cloning it from my github repo, what you get on should be
a folder containing a file named ``setup.py''. Change to this directory
in the terminal and type in:

\begin{Shaded}
\begin{Highlighting}[]
\ExtensionTok{pip}\NormalTok{ install {-}e .}
\end{Highlighting}
\end{Shaded}

\texttt{pip} will read the rules defined in ``setup.py'' and handle all
the dependencies for you. This will install pyADAPT as a package, along
with all the dependencies such as numpy and scipy, to your python
environment in editable mode (what \texttt{-e} stands for). Make sure
you have pip executable from the proper environment included in the path
of the operating system. Python from official website or other
scientific distribution should both work fine.

Then you can try the toy model by running the script ``run\_toy.py''.
Hopefully there's no error.

\hypertarget{dataset}{%
\subsection{DataSet}\label{dataset}}

DataSet class contains the experimental data we obtained from different
sources. It is organized into a list of all the states and fluxes in the
data set. And It is able to produce a spline interpolation evaluated at
different time intervals.

To instantiate DataSet, a raw data and a corresponding data description
file should be provided.

\begin{Shaded}
\begin{Highlighting}[]
\NormalTok{data }\OperatorTok{=}\NormalTok{ DataSet(raw\_data\_path}\OperatorTok{=}\StringTok{\textquotesingle{}data/toyModel/toyData.mat\textquotesingle{}}\NormalTok{,}
\NormalTok{            data\_specs\_path}\OperatorTok{=}\StringTok{\textquotesingle{}data/toyModel/toyData.yaml\textquotesingle{}}\NormalTok{)}
\end{Highlighting}
\end{Shaded}

Since there's no way to standardize the way raw data is stored and
defined, it is also possible for users to first read the raw data and
the data specs as python dictionaries and the use them to instantiate
the data set.

Currently, this DataSet class meets all the needs of the toy model. For
more complicated models which contains more than one measurement, such
as the clamp model with four measurements, class \texttt{DataSets} will
be a future solution.

\hypertarget{model}{%
\subsection{Model}\label{model}}

\texttt{Model} is a abstract class that provide a convenient interface
to define an ADAPT style model. The goal is to keep the definition of
models in both ADAPT and pythonic way. In order to define a new model
\texttt{NewModel}, one should first inherit it from
\texttt{NewModel(Model)}, add parameters, constants, states in
\texttt{\_\_init\_\_} method and extend three important methods:
\texttt{odefunc}, \texttt{reactions} and \texttt{inputs}. Check
\href{../pyADAPT/models/toy_model.py}{\texttt{pyADAPT.models.toy\_model.ToyModel}}
for example.

\hypertarget{odefunc}{%
\subsubsection{odefunc}\label{odefunc}}

\texttt{odefunc} will be passed as an argument of
\texttt{scipy.integrate.solve\_ivp}. There's not much restriction on how
\texttt{odefunc} should be implemented as long as the it takes time,
initial values and parameters as the arguments. The \texttt{Model} class
works by remembering the order of the states and parameters. Thus,
\texttt{odefunc} should always return the derivatives in the same order
as the initial values (\texttt{x0}).

\hypertarget{reactions}{%
\subsubsection{reactions}\label{reactions}}

Reactions are actually any arbitrary expression that uses the components
of the model to obtain a value. In the MATLAB implementation, it can be
either real reaction fluxes or a interesting intermediate value in the
simulation that we would like to keep track of. In some models, the
model outcome is fitted to the experimentally measured fluxes. I will
implement this function when dealing with clamp model.

\hypertarget{inputs}{%
\subsubsection{inputs}\label{inputs}}

Inputs is not useful in the toy model but is important in the clamp
model. It's function is overlapping with the constants in the model, so
further inspection is needed. I will work on it after I learn more about
the clamp model.

\hypertarget{adapt}{%
\subsection{ADAPT}\label{adapt}}

ADAPT class contains all the procedure to perform an ADAPT simulation.
It takes a model instance and a data set instance as the arguments. And
it will calculate the parameter trajectory. It supports parallel
computing with python multiprocessing module. In an ADAPT simulation,
there are usually many iterations, class \texttt{ADAPT} is able to run
multiple iterations at a time by assigning them to different processes.
